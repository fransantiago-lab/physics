\documentclass{book}
\usepackage[spanish,es-nolists,es-tabla, es-nodecimaldot]{babel} % Asegúrate de incluir 'es-nolists' si no quieres que babel altere las listas.


\addto{\captionsspanish}{%
	\renewcommand{\bibname}{Referencias}
}


\usepackage[DIV=14,BCOR=2mm,headinclude=true,footinclude=false]{typearea}
\usepackage{fancyhdr}
\usepackage{subcaption}
\usepackage{multirow}
\usepackage{siunitx}
\fancypagestyle{plain}{
	\fancyhf{}
	\renewcommand{\headrulewidth}{0pt}
	\fancyfoot[C]{\thepage}
}
\pagestyle{fancy}
\fancyhf{}
\fancyfoot[C]{\thepage}

\usepackage{listings}

\usepackage{hhline}
\usepackage{caption}
\captionsetup[figure]{font=small}
\usepackage[table,dvipsnames]{xcolor}
\usepackage{colortbl} % Necesario para \cellcolor
% Paquete para idioma español

% Paquete para personalizar los márgenes
\usepackage[margin=2.5cm]{geometry}
\usepackage{adjustbox}
%Estos dos paquetes son necesarios para escribir cómodamente los
% los documentos en Español, sin necesidad de utilizar secuencias
% de escape como {\'\i} para poner una "í".
\usepackage{titlesec}
\titleformat{\chapter}[display]
{\normalfont\huge\bfseries}{\chaptertitlename\ \thechapter}{20pt}{\Huge}
\titlespacing*{\chapter}{0pt}{0pt}{40pt}
%Para incluir figuras se necesitarán las macros definidas en este paquete
\usepackage{graphicx}

% Para fijar la posición de las imágenes
\usepackage{float}     
\usepackage{xcolor}
\usepackage[table]{xcolor}
%Este paquete permite incluir y resaltar URLs
\usepackage{url}
\usepackage{amsmath, amssymb, bm}
%Este paquete, cuando se usa pdflatex, formatea el documento añadiendo
% enlaces internos entre las referencias y los objetos referenciados
% (por ejemplos, figuras, tablas, referencias de la bibliografía).
\usepackage[unicode=true, pdfusetitle, bookmarks=true, bookmarksnumbered=false, bookmarksopen=false,
breaklinks=true,pdfborder={0 0 1},backref=false,colorlinks=false]{hyperref}

\usepackage{threeparttable}
\usepackage{tabularx}
\usepackage{booktabs}
\newcommand{\ra}[1]{\renewcommand{\arraystretch}{#1}}
% Paquete para personalizar las listas
\usepackage{enumitem} 
\setlist[itemize]{label=-}

%Este paquete puede ser útil si se quieren incluir algunos símbolos
% especiales en las ecuaciones

\usepackage{array}
\usepackage{gensymb}
%Esta definición permite introducir la dirección de los autores
% y mostrarla convenientemente
\newcommand{\address}[1]{\par {\raggedright #1\vspace{1.4em}\noindent\par}}


%Inicio del documento (hasta que se cierre con \end{document}
% Paquete para personalizar los títulos de las secciones
\usepackage{titlesec}
% Paquete para modificar el formato del abstract
\usepackage{abstract}
% Modificar el formato del abstract
\renewcommand{\abstractnamefont}{\normalfont\large\bfseries}
\renewcommand{\abstracttextfont}{\normalfont\normalsize}
\newcolumntype{C}[1]{>{\centering\arraybackslash}p{#1}}
\renewenvironment{abstract}{
\begin{center}
	\bfseries\abstractname\vspace{-\baselineskip}
\end{center}
\list{}{\leftmargin1.5cm \rightmargin\leftmargin}
\item\relax
}{
\endlist
}



% Paquete para ajustar la tabla de contenidos con etoc
\usepackage{etoc}

% Estilo de artículo para cada capítulo
\titleformat{\chapter}[display]{\normalfont\huge\bfseries}{\chaptertitlename\ \thechapter}{20pt}{\Huge}


\makeatletter
\renewenvironment{thebibliography}[1]
{% Utilizamos \section{Referencias} para que aparezca numerado como una sección normal
	\section{Referencias}
	\list{\@biblabel{\@arabic\c@enumiv}}%
	{\settowidth\labelwidth{\@biblabel{#1}}%
		\leftmargin\labelwidth
		\advance\leftmargin\labelsep
		\@openbib@code
		\usecounter{enumiv}%
		\let\p@enumiv\@empty
		\renewcommand\theenumiv{\@arabic\c@enumiv}}%
	\sloppy
	\clubpenalty4000
	\@clubpenalty \clubpenalty
	\widowpenalty4000%
	\sfcode`\.\@m}
{\def\@noitemerr
	{\@latex@warning{Empty `thebibliography' environment}}%
	\endlist}
\makeatother





